\documentclass[a4paper]{article}

% Package imports
\usepackage[utf8]{inputenc}
\usepackage{amsmath}
\usepackage{amsfonts}
\usepackage{amssymb}
\usepackage{graphicx}
\usepackage{hyperref}
\usepackage{geometry}

% Set margins
\geometry{a4paper, margin=1in}

% Title
\title{USE OF BLOCKCHAIN TECHNIQUES IN KENYA FOR INFORMATION DISPLAYING}
\author{Sharon Jepkemboi \\
P2815736 \\
Project Proposal, Planning and Project Management}

% Abstract environment
\newenvironment{abstract}{
    \begin{center}
        \bfseries \abstractname
    \end{center}
    \begin{quote}
}{
    \end{quote}
}

\begin{document}

% Title page
\maketitle

% Abstract
\begin{abstract}
Credit information sharing confirms balance in the lending industry by minimizing the existence of data imbalance across the creditor and clients; it includes a transfer of credit-related data in creditors by Credit Reference Bureaus (CRBs). With the goal of resolving data inequalities issue, blockchain technology enhances reporting capabilities through authenticity, centralization, openness, safety, and dependability. This study examined the issues surrounding credit ratings in Kenya and introduced the use of a blockchain-based structure for an application for expressing credit data.

Recognition and confidence among users when assessing the usability of blockchain-powered debit reporting tools for investors. What is the relationship between lenders' desire to implement blockchain relies on credit monitoring platforms and degree for confidence with the technology's blockchain? Use of blockchain contributes to safe exchange of credit information. Questionnaires and surveys are used to gather quantitative data on creditors' opinions, degrees of trust and ability to embrace the applications also used to gauge creditors' levels of confidence in blockchain technology, as well as their perceptions of its advantages and risks. Finally, target the banks, micro-lending organizations and lending institutions in various regions in Kenya and employ statistical methods (such as the regression model analysis) to find relationships between acceptance desire as well as confidence stages. Restoring confidence within creditors to guarantee personal and financial info stays secure regardless of an open ecosystem. It must abide by national encryption standards. The venture will have significant effects on Kenyan lenders' acceptance and confidence in powered by blockchain credit score devices.
\end{abstract}

\section{Introduction}

\subsection{Background}
Financial institutions and investors in Kenya post data concerning their debts with Credit Reference Bureaus (CRBs) through a process known as Credit Information Sharing (CIS). Firms with a Central Bank of Kenya authorization are known as CRBs. They obtain credit data from various financial institutions and supply it with lenders so they can make informed decisions. CRBs provide notifications to lenders inquiring regarding the borrower's paying history. CRB were introduced in 2010 to share credit data and this system got its start. This has improved the efficiency and transparency of the country's debit in the marketplace.

As per Wang et al. (2019) By releasing an article about Bitcoin in 2008 then seeing it take off in 2013, Satoshi Nakamoto is credited as being the creator of blockchain technology. The Ethereum-based a platform, which ensures the carrying out of agreements among both sides, became available in 2015 and allowed blockchain to interact using loans for leases. According to Blossey et al. (2020) there has been a lot about anticipation surrounding digital wallets that use an online web model to popular digital currencies while it accompanies. Blockchain technologies add data that is replicated in an instant throughout the system and mimics an open database (Priyadarshini, I., 2019). As per Zheng et al. (2022) state that the likelihood of inadequate independent oversight can be decreased by combining safeguarding confidentiality with blockchain technology. An available crucial while a secure critical are generated for each system part by the blockchain's degree of system (Rico-Pena et al., 2023). As stated by Hashem et al. (2020) blockchain technology uses digital encryption to prevent tampering with its record. Based on Zhu et al. (2020) state that the blockchain-based credit rating method lowers risk regarding credit, increases revenues and strengthens costs. In line with Liu et al. (2022) due to its secure verification, confidentiality and additional relevant attributes blockchain remains an excellent platform for occupying credit data.

As stated by Dutta et al. (2020) interconnected online used record is what blockchain technology offers, the fact that information is highly secure for companies to function. As reported by Adusei et al. (2022) there is likely a decrease in the probability of default in the financial sector if scores of credits are present. As mentioned by Bowen et al. (2021) CRBs have been created to evaluate a lending applicant's credibility in order to assist investors in making credit choices more quickly and accurately. As claimed by Zhang et al. (2020) this should be implemented a collaborative blockchain technique that has excellent management. Per Javaid et al. (2022) particularly in light of the lack of skilled developers with blockchain expertise, businesses have to change their present systems in order to set up the technology known as blockchain. In line with Moturi et al. (2020) methods and tactics that incorporate risk mitigation vital have been suggested to boost the effectiveness of managing technology hazards in mobile banking borrowing in safeguard to Kenya's banking environment. As mentioned by Tang et al. (2020) The United Nations considers blockchain to be one of the most promising innovations for equitable growth worldwide (UN, 2018), blockchain-related startup activity is booming worldwide due to technological advances and industries. As a result, advanced uses are being developed, and decision-makers must weigh the possible effects.

\subsection{Research gap}
Following Mariga (2022) claims that he creates a blockchain-based on prototype for the exchange of financial data, the achievement and efficacy of prototype within the finance distributing domain contingent consumer acceptance and confidence. Therefore, it is imperative to assess the user experience (UX) and availability of these networks to guarantee that they satisfy the requirements of financiers. The application will become an accepted instrument in the banking sector as a result of adequate user acceptance guarantees precise risk evaluation, deeper exchange of information and higher market impact. Having trust that financial institutions will make open and well-informed loans that protect the confidentiality of customers. To guarantee a dependable user interface, will make sure it is suitable with the browser they use on the internet.

When contrasted with antiquated reporting techniques the adoption of blockchain technology acknowledge stating systems may dramatically raise the percentage of loan authorization within Kenyan creditors. Based on the blockchain evaluation websites are seen by Kenyan financiers to be increasingly safe and open to the public.

\section{Methodology}
Choosing an accurate reflection of Kenyan loan providers such as finance and micro loans organizations to account for different types of financiers gather information on the acceptance costs of loans before following the implementation of blockchain technology and utilize private data via granting structures. To ascertain whether the findings have statistically meaningful and perform hypothesis validation at 95\% certainty, employ ANOVA to find out when there is exist of significant variation in credit authorization rates among the two-time frames and then evaluate outcomes to comprehend the influence of blockchain on granting credit. To attain saturation of information, incorporate CRBs will create an interview steer about questions that are flexible and cantered on the difficulties they face, views and instances.

Employing a thematic approach to find, evaluate and communicate trends in the info. Examining the main topics to gain insight into lenders' general opinions regarding the blockchain relies credit assessment services, whereas assurance guarantees that customers feel at ease while employing the site Substantial user adoption shows that the platform satisfies the functionality and accessibility desires of lending institutions. Blockchain-based credit monitoring platforms enhance user experience (UX) alongside mobility resulting in increase in financier approval and application. This is achieved using design that prioritizes users’ robust safety protocols and adherence to regulations.

\section{Ethical, legal and social implications}
Lenders must receive complete disclosures concerning the way their details will be handled, utilized, preserved and passed on in order to guarantee adequate protection of their private and business data. To reduce unfairness and prejudice rating methods ought to be open and equitable. Laws pertaining to privacy must be followed by blockchain-based channels on a blockchain system establishing who owns the datastore. Acquiring confidence in the blockchain relies websites requires transparent dissemination of information regarding their advantages, safeguards and moral obligations.

\section{Work plan}
The Gantt chart provides an outline of an assignment time frame with start and finish dates, lengths and tasks that connect. In order to keep the work on schedule it assists with planning, allocating resources and tracking advancement towards time constraints. The dependencies of tasks and crucial path evaluation, which determines the least amount of time needed to finish the task at hand are other key components of Program Evaluation Review Technique (PERT) this allows for recognizing essential duties that if rescheduled could affect the task's time frame, manage unpredictability and optimize the arrangement of tasks.

\section{Conclusion}
The debt consolidation marketplace has an increase in securities from organizations over the past five decades, as well as the expansion of internet lenders whose employ credit histories from CRBs to make loan offers has significantly increased marginalized individuals' access to funds. It is crucial to figure out how to use blockchain systems to preserve the accuracy of data in order to guarantee the borrowing consumer future expansion and viability. The knowledge acquired will help make well-informed decisions, encourage broad adoption and guarantee the successful implementation of blockchain. Banking unions, finance companies and microfinance associations will gain from recognizing that blockchain boosts the safety of information, lowers deception and boosts credit monitoring. By systematically integrating technologies like blockchain, such companies will increase productivity and consumer trust. This may be achieved through an awareness of the elements that influence acceptance and confidence. The results of this study will improve operational effectiveness and boost client trust by advancing safe and effective credit rating platforms.

\section{References}
\begin{enumerate}
    \item Adusei, M. and Adeleye, N., (2022). Credit information sharing and non‐performing loans: the moderating role of creditor rights protection. Available at: \url{https://onlinelibrary.wiley.com/doi/full/10.1002/ijfe.2398?casa_token=DmQjByrqUSwAAAAA\%3AaWRqyHcW6tGb2iZuCppbxMPcCiCiNWDAhMmVmIErA3gb-rDu36mitk_a_JMnKS1BhOBFRYyUQ3EvhFk}. Accessed on: 29/5/2024.
    \item Bowen, N.J. and Makokha, E.N., (2021). Effects of Credit Information Sharing on Performance of Savings and Credit Cooperative Societies in Kenya. Available at: \url{https://www.paperpublications.org/upload/book/paperpdf-1614250158.pdf}. Accessed on: 29/5/2024.
    \item Blossey, G., Eisenhardt, J. and Hahn, G., (2019). Blockchain technology in supply chain management: An application perspective. Available at: \url{https://aisel.aisnet.org/cgi/viewcontent.cgi?article=1838&context=hicss-52}. Accessed on: 30/5/2024.
    \item Dutta, P., Choi, T.M., Somani, S. and Butala, R., (2020). Blockchain technology in supply chain operations: Applications, challenges and research opportunities. Available at: \url{https://www.sciencedirect.com/science/article/pii/S266660302100021X#sec1}. Accessed on: 30/5/2024.
    \item Hashemi Joo, M., Nishikawa, Y. and Dandapani, K., (2020). Cryptocurrency, a successful application of blockchain technology. Available at: \url{https://www.emerald.com/insight/content/doi/10.1108/MF-09-2018-0451/full/html#sec001}. Accessed on: 30/5/2024.
    \item Javaid, M., Haleem, A., Singh, R.P., Suman, R. and Khan, S., (2022). A review of Blockchain Technology applications for financial services. Available at: \url{https://www.sciencedirect.com/science/article/pii/S2772485922000606#sec9}. Accessed on: 29/5/2024.
    \item Liu, C.Y., Dong, T.Y. and Meng, L.X., (2022). Cross-Border Credit Information Sharing Mechanism and Legal Countermeasures Based on Blockchain 3.0. Available at: \url{https://www.hindawi.com/journals/misy/2022/6972647/}. Accessed on: 30/5/2024.
    \item Moturi, C. and Ogoti, G., (2020). Strengthening technology risk management in mobile money lending. Available at: \url{https://www.inderscienceonline.com/doi/abs/10.1504/IJFSM.2020.111105}. Accessed on: 30/5/2024.
    \item Mariga, J., (2022). An Application of the Blockchain Technology in Credit Information Sharing in Kenya. Available at: \url{https://www.inderscienceonline.com/doi/abs/10.1504/IJBC.2023.132707}. Accessed on: 3/6/2024.
    \item Priyadarshini, I., (2019). Introduction to blockchain technology. Available at: \url{https://scholar.google.com/scholar?hl=en&as_sdt=0\%2C5&q=Introduction+to+blockchain+technology&btnG}. Accessed on: 28/5/2024.
    \item Rico-Pena, J.J., Arguedas-Sanz, R. and Lopez-Martin, C., (2023). Models used to characterise blockchain features. A systematic literature review and bibliometric analysis. Available at: \url{https://scholar.google.com/scholar?q=features+of+blockchain&hl=en&as_sdt=0\%2C5&scioq=blockchain+technology+for++credit+sharing+information&as_ylo=2019&as_yhi=2024}. Accessed on: 29/5/2024.
    \item Tang, Y., Xiong, J., Becerril-Arreola, R. and Iyer, L., (2020). Ethics of blockchain: A framework of technology, applications, impacts, and research directions. Available at: \url{https://www.emerald.com/insight/content/doi/10.1108/ITP-10-2018-0491/full/html}. Accessed on: 30/5/2024.
    \item Wang, Y., Han, J.H. and Beynon-Davies, P., (2019). Understanding blockchain technology for future supply chains: a systematic literature review and research agenda. Available at: \url{https://www.emerald.com/insight/content/doi/10.1108/SCM-03-2018-0148/full/pdf?title=understanding-blockchain-technology-for-future-supply-chains-a-systematic-literature-review-and-research-agenda}. Accessed on: 29/5/2024.
    \item Wu, B. and Duan, T., (2019), March. The application of blockchain technology in financial markets. Available at: \url{https://iopscience.iop.org/article/10.1088/1742-6596/1176/4/042094/pdf}. Accessed on: 29/5/2024.
    \item Zheng, K., Zheng, L.J., Gauthier, J., Zhou, L., Xu, Y., Behl, A. and Zhang, J.Z., (2022). Blockchain technology for enterprise credit information sharing in supply chain finance. Available at: \url{https://scholar.google.com/scholar?q=related:Ni4-lSabpyAJ:scholar.google.com/&scioq=blockchain+technology+for++credit+sharing+information&hl=en&as_sdt=0\%2C5&as_ylo=2019&as_yhi=2024}. Accessed on: 29/5/2024.
    \item Zhang, J., Tan, R., Su, C. and Si, W., (2020). Design and application of a personal credit information sharing platform based on consortium blockchain. Available at: \url{https://www.sciencedirect.com/science/article/pii/S2214212620308139?casa_token=msTIByr2w60AAAAA:OahFfIOb44gHYc8h5FPXZZ2SPztWNnCTboflBQ8Y9cQwRFIufWlFu3hb14HWvNjeN9fWTjxnZg#sec3}. Accessed on: 29/5/2024.
    \item Zhu, X., (2020). Blockchain-based identity authentication and intelligent Credit reporting. Available at: \url{https://iopscience.iop.org/article/10.1088/1742-6596/1437/1/012086/pdf}. Accessed on: 30/5/2024.
\end{enumerate}

\end{document}
