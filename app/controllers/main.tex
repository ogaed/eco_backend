\documentclass[a4paper, 12pt]{article}

% Package imports
\usepackage[utf8]{inputenc}
\usepackage{amsmath}
\usepackage{amsfonts}
\usepackage{amssymb}
\usepackage{graphicx}
\usepackage{hyperref}

% Title
\title{USE OF BLOCKCHAIN TECHNIQUES IN KENYA FOR INFORMATION DISPLAYING}
\author{Sharon Jepkemboi \\
P2815736 \\
Project Proposal, Planning and Project Management}

% Abstract environment
\newenvironment{abstract}{
    \begin{center}
        \bfseries \abstractname
    \end{center}
    \begin{quote}
}{
    \end{quote}
}

\begin{document}

% Title page
\maketitle

% Abstract
\begin{abstract}
Credit information sharing confirms balance in the lending industry by minimizing the existence of data
imbalance across the creditor and clients; it includes a transfer of credit-related data in creditors by Credit
Reference Bureaus (CRBs). With the goal of resolving data inequalities issue, blockchain technology
enhances reporting capabilities through authenticity, centralization, openness, safety, and dependability.
This study examined the issues surrounding credit ratings in Kenya and introduced the use of a
blockchain-based structure for an application for expressing credit data.

Recognition and confidence among users when assessing the usability of blockchain-powered debit
reporting tools for investors. What is the relationship between lenders' desire to implement blockchain
relies on credit monitoring platforms and degree for confidence with the technology's blockchain? Use of
blockchain contributes to safe exchange of credit information. Questionnaires and surveys are used to
gather quantitative data on creditors' opinions, degrees of trust and ability to embrace the applications also
used to gauge creditors' levels of confidence in blockchain technology, as well as their perceptions of its
advantages and risks. Finally, target the banks, micro-lending organizations and lending institutions in
various regions in Kenya and employ statistical methods (such as the regression model analysis) to find
relationships between acceptance desire as well as confidence stages. Restoring confidence within
creditors to guarantee personal and financial info stays secure regardless of an open ecosystem. It must
abide by national encryption standards. The venture will have significant effects on Kenyan lenders'
acceptance and confidence in powered by blockchain credit score devices.
\end{abstract}

% Sections (placeholders for further content)
\section{Introduction}
% Your introduction content here

\section{Literature Review}
% Your literature review content here

\section{Methodology}
% Your methodology content here

\section{Findings and Discussion}
% Your findings and discussion content here

\section{Conclusion}
% Your conclusion content here

% References (example using bibtex)
\begin{thebibliography}{9}
    \bibitem{example1} Author, A. (Year). Title of the paper. \textit{Journal Name}, Volume(Issue), Page numbers.
    \bibitem{example2} Author, B. (Year). \textit{Title of the book}. Publisher.
\end{thebibliography}

\end{document}
